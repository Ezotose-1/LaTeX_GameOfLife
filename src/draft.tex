\documentclass[border=1mm,multi=page]{standalone}

\usepackage{tikz}
\usepackage{xinttools}

\begin{document}

\def\initialboard{
000000000000000000000000000100000000,
000000000000000000000000001010000000,
000000000110000000000000001101000000,
000000000101000000000000001101100011,
000011000000100000000000001101000011,
110100100100100000000000001010000000,
110011000000100000000100000100000000,
000000000101000000010100000000000000,
000000000110000000001100000000000000,
000000000000000000000000000000000000,
000000000000000000000000000000000000,
000000000000000000000000000000000000,
000000000000000000000000000000000000,
000000000000000000000000000010000000,
000000000000000000000000000001000000,
000000000000000000000000000111000000,
000000000000000000000000000000000000,
000000000000000000000000000000000000,
000000000000000000000000000000000000
}

\newcount\Xmax
\newcount\Ymax
\newcount\NeighboursCount

\Ymax 0

% \brief create variables for each cell of the board %
\xintFor #1 in \initialboard  \do
{%
    \advance\Ymax by 1

    \Xmax 0

    \xintFor* #2 in {#1} \do
    {%
        \advance\Xmax by 1
        \expandafter\def\csname node\the\Xmax.\the\Ymax\endcsname {#2}%
    }%
}%


% Create OUT-OF-BOUND node %
% 0 -> Xmax %
\xintFor #1 in \xintintegers \do
{%
    \expandafter\def\csname oldnode\the#1.0\endcsname {0}%
    \expandafter\def\csname oldnode\the#1.\the\numexpr\Ymax+1\endcsname {0}%
    % End of column -> break %
    \ifnum #1=\Xmax\expandafter\xintBreakFor\fi
}%
% 0 -> Ymax %
\xintFor #2 in \xintintegers \do
{%
    \expandafter\def\csname oldnode0.\the#2\endcsname {0}%
    \expandafter\def\csname oldnode\the\numexpr\Xmax+1.\the#2\endcsname {0}%
    % End of column -> break %
    \ifnum #2=\Ymax\expandafter\xintBreakFor\fi
}%
% OUT OF BOARD angles %
\expandafter\def\csname oldnode0.0\endcsname {0}%
\expandafter\def\csname oldnode\the\numexpr\Xmax+1.0\endcsname {0}%
\expandafter\def\csname oldnode0.\the\numexpr\Ymax+1\endcsname {0}%
\expandafter\def\csname oldnode\the\numexpr\Xmax+1.\the\numexpr\Ymax+1\endcsname {0}%

% \brief For each current node -> create a save node to build the next generations %
\newcommand\SaveNodes{%
    \xintFor ##1 in \xintintegers \do
    {%
        \xintFor ##2 in \xintintegers \do
        {%
        	\ifcase\csname node\the##1.\the##2\endcsname
            	\expandafter\def\csname oldnode\the##1.\the##2\endcsname {0}%
            \else
            	\expandafter\def\csname oldnode\the##1.\the##2\endcsname {1}%
            \fi
            % End of line -> break %
           \ifnum ##2=\Ymax\expandafter\xintBreakFor\fi
       }%
       % End of column -> break %
       \ifnum ##1=\Xmax\expandafter\xintBreakFor\fi
   }%
}%

% \brief Pretty print the current generation %
\newcommand\PPrint {%
    \begin{page}
    \begin{tikzpicture}
        % Draw the main grid %
        \draw(0,0) grid(\the\Xmax,\the\Ymax);
        % Draw the right living cell %
        \xintFor ##1 in \xintintegers \do
        {%
            \xintFor ##2 in \xintintegers \do
            {%
                \ifcase\csname node\the##1.\the##2\endcsname
                \else
                    \fill(\the##1 - 1, \the##2 - 1)+(0.5, 0.5) circle (0.4);
                \fi
                % End of line -> break %
                \ifnum ##2=\Ymax\expandafter\xintBreakFor\fi
            }%
            % End of column -> break %
            \ifnum ##1=\Xmax\expandafter\xintBreakFor\fi
        }%
    \end{tikzpicture}
    \end{page}
}%

% #1 = x, #2 = y %
\newcommand\Neighborhood[2]{%

    \NeighboursCount 0

    \advance\NeighboursCount by \csname oldnode\number#1.\number\numexpr#2+1\endcsname
    \advance\NeighboursCount by \csname oldnode\number\numexpr#1+1.\number\numexpr#2+1\endcsname
    \advance\NeighboursCount by \csname oldnode\number\numexpr#1-1.\number\numexpr#2+1\endcsname
    \advance\NeighboursCount by \csname oldnode\number\numexpr#1+1.\number#2\endcsname
    \advance\NeighboursCount by \csname oldnode\number\numexpr#1-1.\number#2\endcsname
    \advance\NeighboursCount by \csname oldnode\number#1.\number\numexpr#2-1\endcsname
    \advance\NeighboursCount by \csname oldnode\number\numexpr#1+1.\number\numexpr#2-1\endcsname
    \advance\NeighboursCount by \csname oldnode\number\numexpr#1-1.\number\numexpr#2-1\endcsname

    %\the\NeighboursCount

}%


\newcommand\ApplyRules[2]{%

	\Neighborhood{\number#1}{\number#2}

	\ifnum \the\NeighboursCount=3
		\expandafter\def\csname node#1.#2\endcsname {1}%
	\else
		\ifnum \the\NeighboursCount=2 
			\ifcase\csname oldnode\number#1.\number#2\endcsname
				\expandafter\def\csname node#1.#2\endcsname {0}%
			\else
				\expandafter\def\csname node#1.#2\endcsname {1}%
			\fi
		\else
			\expandafter\def\csname node#1.#2\endcsname {0}%
		\fi
	\fi

}%



% Generate the new node value for each saveNodes %
\newcommand\Generate {%
    \xintFor ##1 in \xintintegers \do
    {%
        \xintFor ##2 in \xintintegers \do
        {%

       		\ApplyRules{\the##1}{\the##2}
            % End of line -> break %
        	\ifnum ##2=\Ymax\expandafter\xintBreakFor\fi
       }%
       % End of column -> break %
       \ifnum ##1=\Xmax\expandafter\xintBreakFor\fi
   }%
}%

\PPrint{}
\SaveNodes{}
%\expandafter\def\csname node1.1\endcsname {1}%
\Generate{}
%\PPrint{}

%\ApplyRules{21}{7}
%\Neighborhood{21}{7}
%\the\NeighboursCount

%\SaveNodes{}
%\Generate{}
\PPrint{}

\end{document}
