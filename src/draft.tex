%
% SPDX-License-Identifier: GPL-2.0-or-later
%
% Latex Game Of Life
%
% Copyright (C) 2022 Pierre B <https://github.com/Ezotose-1)
%
% This program is free software; you can redistribute it and/or modify it
% under the terms of the GNU General Public License as published by the Free
% Software Foundation; either version 2 of the License, or (at your option)
% any later version.
%
% This program is distributed in the hope that it will be useful, but WITHOUT
% ANY WARRANTY; without even the implied warranty of MERCHANTABILITY or
% FITNESS FOR A PARTICULAR PURPOSE. See the GNU General Public License for
% more details.
%
% You should have received a copy of the GNU General Public License along with
% this program; if not, write to the Free Software Foundation, Inc., 51
% Franklin Street, Fifth Floor, Boston, MA 02110-1301, USA.
%

\documentclass[border=1mm,multi=page]{standalone}

\usepackage{tikz}
\usepackage{xinttools}

\begin{document}

% Load board of the game : 0 deads cells, 1 alives cells %
% GLIDER GUN %
\def\initialboard{
    000000000000000000000000000100000000,
    000000000000000000000000001010000000,
    000000000110000000000000001101000000,
    000000000101000000000000001101100011,
    000011000000100000000000001101000011,
    110100100100100000000000001010000000,
    110011000000100000000100000100000000,
    000000000101000000010100000000000000,
    000000000110000000001100000000000000,
    000000000000000000000000000000000000,
    000000000000000000000000000000000000,
    000000000000000000000000000000000000,
    000000000000000000000000000000000000,
    000000000000000000000000000010000000,
    000000000000000000000000000001000000,
    000000000000000000000000000111000000,
    000000000000000000000000000000000000,
    000000000000000000000000000000000000,
    000000000000000000000000000000000000
}

% % PULSAR %
\def\initialboard {
    00000000000000000000,
    00000000000000000000,
    00000000000000000000,
    00000111000111000000,
    00000000000000000000,
    00010000101000010000,
    00010000101000010000,
    00010000101000010000,
    00000111000111000000,
    00000000000000000000,
    00000111000111000000,
    00010000101000010000,
    00010000101000010000,
    00010000101000010000,
    00000000000000000000,
    00000111000111000000,
    00000000000000000000,
    00000000000000000000,
    00000000000000000000,
    00000000000000000000
}

% % RANDOM CELLS %
\def\initialboard{
    00000000000000000000000000000000000000000000000000,
    00000000000000000000000000000000000000000011000000,
    00000000000000000000000000000000000000000010100000,
    00000000000000000000000000000000000000000001000000,
    00000000000000000000000000000000000000111000000011,
    00000000000000000000000000000000000000000000000011,
    00000000000000000000000000000000000010000010000000,
    00000000000000000000000000000000000010000010000000,
    00000000000000000000000000000000000010000010000000,
    00000000000000000000000000000000000000000000000000,
    00000000000000000000000000000000000000111000000000,
    00000000000000000000000000000000000000000000000000,
    00000000000000000000000000000000000000000000000000,
    00000000000000000000000000000000000000000000000000,
    00100000000000000000000000000000000000000000000000,
    01010000000000000000000000000000000000000000000000,
    10100000000000110000000000011000000000000000000000,
    01000000000000110000000001001000000000000000000000,
    00000000000000000000000001110000000000000000000000,
    00000000000000000000000000000000000000000000000000
}

% % FAT RANDOM SCHEMA %
\def\initialboard{
    10000000001001000000001000000000000000000000011000,
    00000001001000100001101100111000000000000000110101,
    00000010110001000001101000100100000000000000110010,
    00000010101001000001011001000100001100000000000000,
    00000001011001000111100001001000010010000000000000,
    00000000000010000100000000000000001001000000000000,
    00000000000000000111000000110000001001000000000000,
    00000000000000000010000000000000001010000000000000,
    00001110000000000010000110000000000100000000000000,
    00000000000000001111001010000000000000000000000000,
    00000000000000001100100100000000000000000000000000,
    00000011100000001100000100000000001000000000000000,
    00000011100000000000110000000000010100000000000000,
    00000100101100000011010000000000010010000000000000,
    00000011011000000001100000000000001100000000000000,
    00000011111000000001000000000000000000000000000000,
    00000001001000000000000000000000000000001100000000,
    00000000110000000000000000000000000000000110000000,
    00000000110000000000000000000000000000000110000000,
    00000000000000000000000000000000000000001110000000,
    00000000000000000000000000000000000000001100000000,
    00000000000000000000000000000000000001011000000111,
    10000000000000000000000000000000000000110000001000,
    10000001100000000000000000000000000000000000001101,
    00000010010000000000000000000000000000000000000000,
    00000001100000000000000000000000000001100000000000,
    00000000000000000000000000000000000010100000000000,
    00000000000000000000000000000000000000100000000000,
    10000000000000000000000000000000000000000000000111,
    11000000011000000000000000000000000000000000001011,
    01100000011000000000000000000000000000000000011001,
    01000000000000000000000000000000101011000000011000,
    10000000000000000000000000000000101010000000001100,
    00000000000000000000000000000000100001011000000000,
    00000000000000000000000000000000010001111000000000,
    00000000000000000000000000000000001110011000000000,
    00000000000000000000000000000000000000000000000000,
    00000000000000111000000000000000000000000100000000,
    00000000000000000000000000001110000000000100000000,
    00000000000010000000000000010101101000000100000000,
    00000000000010000000000000100000000100000000000000,
    00000000000010000000000001001111000100000000000000,
    00000000000000000000000001110000001000000000000000,
    00000000000000000000000000000000000000000000000110,
    00000000000000000110000000000000000000000000001010,
    00000000000011001000000000000000000000000000010010,
    00000000000010100100000000000000100000000000010000,
    00000000000010000001000000000001001000000000110100,
    00000000000000000010000000000000110000000000000011,
    00000000000101000010010000000000000000000000011101
}

% % RANDOMS CELL + FEW SHEMAS %
\def\initialboard{
    00000000000000000000000000000000000000000000000000000000000000000000000000000000,
    00000000000000000000000000000000000000000000000000000000000000000000000000000000,
    00000000000000001001110001000000001000000000000000000010000000000000000000000000,
    00000000000000000100110000000000000000000000000000000000000000000110000001000000,
    00000000000000000011110000010000000000000000000000000000000000010010000001000000,
    00000000000000000000000111010000000000000000000000000000000100110100000001000000,
    00000000000000000000001110110000000000000000000000000000001010010000000000000000,
    00000000000000001000000001000000000000000000000000000000000001000000000000000000,
    00000000000000110110001111000000000000000000000000000000001101000000000000000000,
    00000000000000111111000011000000000000000000000000000000000010100000000000000000,
    00000000000000010010000000000000000000000000000000000000101110110000011100000000,
    00000000000000001100000000000000000000000000000000000000000010100000000000000000,
    00000000000000000000001000000000000000000000000000000000100001100000000000000000,
    00000000000000000000001000000000000000000000000000000000001010000000000000000000,
    00000000000000000000101110000000010100000000000000000010000011000000000000000000,
    00000000000000000010010000000000001000000000000000000000000010000000000000000000,
    00000000000000000000101110000000010100000000000000000010000000000000000000011100,
    00000000000000000000000000000000000000000000000000000000000000000000000000000000,
    00000000000000000000000000000000000000000000000000000000000000000000000000000000
}


% \input{../ref/}


% Create and init globals variables %
\newcount\Xmax
\newcount\Ymax
\newcount\NeighboursCount

\Ymax 0

% \brief: Create variables for each cell of the board %
\xintFor #1 in \initialboard  \do
{%
    \advance\Ymax by 1

    \Xmax 0

    \xintFor* #2 in {#1} \do
    {%
        \advance\Xmax by 1
        \expandafter\def\csname node\the\Xmax.\the\Ymax\endcsname {#2}%
    }%
}%

% \brief: Create OUT-OF-BOUND nodes to avoid Neighborhood errors %
% 1 -> Xmax %
\xintFor #1 in \xintintegers \do
{%
    \expandafter\def\csname oldnode\the#1.0\endcsname {0}%
    \expandafter\def\csname oldnode\the#1.\the\numexpr\Ymax+1\endcsname {0}%
    % End of column -> break %
    \ifnum #1=\Xmax\expandafter\xintBreakFor\fi
}%
% 1 -> Ymax %
\xintFor #2 in \xintintegers \do
{%
    \expandafter\def\csname oldnode0.\the#2\endcsname {0}%
    \expandafter\def\csname oldnode\the\numexpr\Xmax+1.\the#2\endcsname {0}%
    % End of column -> break %
    \ifnum #2=\Ymax\expandafter\xintBreakFor\fi
}%
% OUT OF BOARD angles %
\expandafter\def\csname oldnode0.0\endcsname {0}%
\expandafter\def\csname oldnode\the\numexpr\Xmax+1.0\endcsname {0}%
\expandafter\def\csname oldnode0.\the\numexpr\Ymax+1\endcsname {0}%
\expandafter\def\csname oldnode\the\numexpr\Xmax+1.\the\numexpr\Ymax+1\endcsname {0}%

% \brief: For each current node -> create a save node to build the next generations %
\newcommand\SaveNodes{%
    \xintFor ##1 in \xintintegers \do
    {%
        \xintFor ##2 in \xintintegers \do
        {%
            \ifcase\csname node\the##1.\the##2\endcsname
                \expandafter\def\csname oldnode\the##1.\the##2\endcsname {0}%
            \else
                \expandafter\def\csname oldnode\the##1.\the##2\endcsname {1}%
            \fi
            % End of line -> break %
           \ifnum ##2=\Ymax\expandafter\xintBreakFor\fi
       }%
       % End of column -> break %
       \ifnum ##1=\Xmax\expandafter\xintBreakFor\fi
   }%
}%

% \brief: Pretty print the current generation of the nodes %
\newcommand\PPrint {%
    \begin{page}
    \begin{tikzpicture}
        % Draw the main grid %
        \draw(0,0) grid(\the\Xmax,\the\Ymax);
        % Draw the right living cell %
        \xintFor ##1 in \xintintegers \do
        {%
            \xintFor ##2 in \xintintegers \do
            {%
                \ifcase\csname node\the##1.\the##2\endcsname
                \else
                    \fill(\the##1 - 1, \the##2 - 1)+(0.5, 0.5) circle (0.4);
                \fi
                % End of line -> break %
                \ifnum ##2=\Ymax\expandafter\xintBreakFor\fi
            }%
            % End of column -> break %
            \ifnum ##1=\Xmax\expandafter\xintBreakFor\fi
        }%
    \end{tikzpicture}
    \end{page}
}%

% \brief: Count all neighbours of a cell and store the result in NeighboursCount %
% \params: #1 is x position, #2 is y position %
\newcommand\Neighborhood[2]{%

    \NeighboursCount 0

    \advance\NeighboursCount by \csname oldnode\number#1.\number\numexpr#2+1\endcsname
    \advance\NeighboursCount by \csname oldnode\number\numexpr#1+1.\number\numexpr#2+1\endcsname
    \advance\NeighboursCount by \csname oldnode\number\numexpr#1-1.\number\numexpr#2+1\endcsname
    \advance\NeighboursCount by \csname oldnode\number\numexpr#1+1.\number#2\endcsname
    \advance\NeighboursCount by \csname oldnode\number\numexpr#1-1.\number#2\endcsname
    \advance\NeighboursCount by \csname oldnode\number#1.\number\numexpr#2-1\endcsname
    \advance\NeighboursCount by \csname oldnode\number\numexpr#1+1.\number\numexpr#2-1\endcsname
    \advance\NeighboursCount by \csname oldnode\number\numexpr#1-1.\number\numexpr#2-1\endcsname

}%

% \brief: Apply GameOfLife rules on one cells %
% \params: #1 is x position, #2 is y position %
\newcommand\ApplyRules[2]{%

    \Neighborhood{\number#1}{\number#2}

    \ifnum \the\NeighboursCount=3
        \expandafter\def\csname node#1.#2\endcsname {1}%
    \else
        \ifnum \the\NeighboursCount=2 
            \ifcase\csname oldnode\number#1.\number#2\endcsname
                \expandafter\def\csname node#1.#2\endcsname {0}%
            \else
                \expandafter\def\csname node#1.#2\endcsname {1}%
            \fi
        \else
            \expandafter\def\csname node#1.#2\endcsname {0}%
        \fi
    \fi

}%



% \brief: Generate the new node value for each saveNodes based on ApplyRules %
\newcommand\Generate {%
    \xintFor ##1 in \xintintegers \do
    {%
        \xintFor ##2 in \xintintegers \do
        {%

            \ApplyRules{\the##1}{\the##2}
            % End of line -> break %
            \ifnum ##2=\Ymax\expandafter\xintBreakFor\fi
       }%
       % End of column -> break %
       \ifnum ##1=\Xmax\expandafter\xintBreakFor\fi
   }%
}%

% \brief: Generate many generations %
% \param: #1 number of generations to build %
\newcommand\Generations[1] {%
    \newcount\X
    \X=#1
    \loop
    \advance \X by -1
    \ifnum \X>-1
        \SaveNodes{}
        \Generate{}
        \PPrint{}
    \repeat

}%

\PPrint{}
% Load precise number of generations into the pdf %
\Generations{100}

\end{document}
