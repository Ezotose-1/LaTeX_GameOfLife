\documentclass[border=1mm,multi=page]{standalone}

\usepackage{tikz}
\usepackage{xinttools}

\begin{document}

% Board of the game : 0 deads cells, 1 alives cells %
\def\initialboard{
000000000000000000000000000100000000,
000000000000000000000000001010000000,
000000000110000000000000001101000000,
000000000101000000000000001101100011,
000011000000100000000000001101000011,
110100100100100000000000001010000000,
110011000000100000000100000100000000,
000000000101000000010100000000000000,
000000000110000000001100000000000000,
000000000000000000000000000000000000,
000000000000000000000000000000000000,
000000000000000000000000000000000000,
000000000000000000000000000000000000,
000000000000000000000000000010000000,
000000000000000000000000000001000000,
000000000000000000000000000111000000,
000000000000000000000000000000000000,
000000000000000000000000000000000000,
000000000000000000000000000000000000
}

% Create and init globals variables %
\newcount\Xmax
\newcount\Ymax
\newcount\NeighboursCount

\Ymax 0

% \brief: Create variables for each cell of the board %
\xintFor #1 in \initialboard  \do
{%
    \advance\Ymax by 1

    \Xmax 0

    \xintFor* #2 in {#1} \do
    {%
        \advance\Xmax by 1
        \expandafter\def\csname node\the\Xmax.\the\Ymax\endcsname {#2}%
    }%
}%

% \brief: Create OUT-OF-BOUND nodes to avoid Neighborhood errors %
% 1 -> Xmax %
\xintFor #1 in \xintintegers \do
{%
    \expandafter\def\csname oldnode\the#1.0\endcsname {0}%
    \expandafter\def\csname oldnode\the#1.\the\numexpr\Ymax+1\endcsname {0}%
    % End of column -> break %
    \ifnum #1=\Xmax\expandafter\xintBreakFor\fi
}%
% 1 -> Ymax %
\xintFor #2 in \xintintegers \do
{%
    \expandafter\def\csname oldnode0.\the#2\endcsname {0}%
    \expandafter\def\csname oldnode\the\numexpr\Xmax+1.\the#2\endcsname {0}%
    % End of column -> break %
    \ifnum #2=\Ymax\expandafter\xintBreakFor\fi
}%
% OUT OF BOARD angles %
\expandafter\def\csname oldnode0.0\endcsname {0}%
\expandafter\def\csname oldnode\the\numexpr\Xmax+1.0\endcsname {0}%
\expandafter\def\csname oldnode0.\the\numexpr\Ymax+1\endcsname {0}%
\expandafter\def\csname oldnode\the\numexpr\Xmax+1.\the\numexpr\Ymax+1\endcsname {0}%

% \brief: For each current node -> create a save node to build the next generations %
\newcommand\SaveNodes{%
    \xintFor ##1 in \xintintegers \do
    {%
        \xintFor ##2 in \xintintegers \do
        {%
            \ifcase\csname node\the##1.\the##2\endcsname
                \expandafter\def\csname oldnode\the##1.\the##2\endcsname {0}%
            \else
                \expandafter\def\csname oldnode\the##1.\the##2\endcsname {1}%
            \fi
            % End of line -> break %
           \ifnum ##2=\Ymax\expandafter\xintBreakFor\fi
       }%
       % End of column -> break %
       \ifnum ##1=\Xmax\expandafter\xintBreakFor\fi
   }%
}%

% \brief: Pretty print the current generation of the nodes %
\newcommand\PPrint {%
    \begin{page}
    \begin{tikzpicture}
        % Draw the main grid %
        \draw(0,0) grid(\the\Xmax,\the\Ymax);
        % Draw the right living cell %
        \xintFor ##1 in \xintintegers \do
        {%
            \xintFor ##2 in \xintintegers \do
            {%
                \ifcase\csname node\the##1.\the##2\endcsname
                \else
                    \fill(\the##1 - 1, \the##2 - 1)+(0.5, 0.5) circle (0.4);
                \fi
                % End of line -> break %
                \ifnum ##2=\Ymax\expandafter\xintBreakFor\fi
            }%
            % End of column -> break %
            \ifnum ##1=\Xmax\expandafter\xintBreakFor\fi
        }%
    \end{tikzpicture}
    \end{page}
}%

% \brief: Count all neighbours of a cell and store the result in NeighboursCount %
% \params: #1 is x position, #2 is y position %
\newcommand\Neighborhood[2]{%

    \NeighboursCount 0

    \advance\NeighboursCount by \csname oldnode\number#1.\number\numexpr#2+1\endcsname
    \advance\NeighboursCount by \csname oldnode\number\numexpr#1+1.\number\numexpr#2+1\endcsname
    \advance\NeighboursCount by \csname oldnode\number\numexpr#1-1.\number\numexpr#2+1\endcsname
    \advance\NeighboursCount by \csname oldnode\number\numexpr#1+1.\number#2\endcsname
    \advance\NeighboursCount by \csname oldnode\number\numexpr#1-1.\number#2\endcsname
    \advance\NeighboursCount by \csname oldnode\number#1.\number\numexpr#2-1\endcsname
    \advance\NeighboursCount by \csname oldnode\number\numexpr#1+1.\number\numexpr#2-1\endcsname
    \advance\NeighboursCount by \csname oldnode\number\numexpr#1-1.\number\numexpr#2-1\endcsname

}%

% \brief: Apply GameOfLife rules on one cells %
% \params: #1 is x position, #2 is y position %
\newcommand\ApplyRules[2]{%

    \Neighborhood{\number#1}{\number#2}

    \ifnum \the\NeighboursCount=3
        \expandafter\def\csname node#1.#2\endcsname {1}%
    \else
        \ifnum \the\NeighboursCount=2 
            \ifcase\csname oldnode\number#1.\number#2\endcsname
                \expandafter\def\csname node#1.#2\endcsname {0}%
            \else
                \expandafter\def\csname node#1.#2\endcsname {1}%
            \fi
        \else
            \expandafter\def\csname node#1.#2\endcsname {0}%
        \fi
    \fi

}%



% \brief: Generate the new node value for each saveNodes based on ApplyRules %
\newcommand\Generate {%
    \xintFor ##1 in \xintintegers \do
    {%
        \xintFor ##2 in \xintintegers \do
        {%

            \ApplyRules{\the##1}{\the##2}
            % End of line -> break %
            \ifnum ##2=\Ymax\expandafter\xintBreakFor\fi
       }%
       % End of column -> break %
       \ifnum ##1=\Xmax\expandafter\xintBreakFor\fi
   }%
}%

% \brief: Generate many generations %
% \param: #1 number of generations to build %
\newcommand\Generations[1] {%
    \newcount\X
    \X=#1
    \loop
    \advance \X by -1
    \ifnum \X>-1
        \SaveNodes{}
        \Generate{}
        \PPrint{}
    \repeat

}%

\PPrint{}
% Load precise number of generations into the pdf %
\Generations{100}

\end{document}
